\PassOptionsToPackage{unicode=true}{hyperref} % options for packages loaded elsewhere
\PassOptionsToPackage{hyphens}{url}
%
\documentclass[]{article}
\usepackage{lmodern}
\usepackage{amssymb,amsmath}
\usepackage{ifxetex,ifluatex}
\usepackage{fixltx2e} % provides \textsubscript
\ifnum 0\ifxetex 1\fi\ifluatex 1\fi=0 % if pdftex
  \usepackage[T1]{fontenc}
  \usepackage[utf8]{inputenc}
  \usepackage{textcomp} % provides euro and other symbols
\else % if luatex or xelatex
  \usepackage{unicode-math}
  \defaultfontfeatures{Ligatures=TeX,Scale=MatchLowercase}
\fi
% use upquote if available, for straight quotes in verbatim environments
\IfFileExists{upquote.sty}{\usepackage{upquote}}{}
% use microtype if available
\IfFileExists{microtype.sty}{%
\usepackage[]{microtype}
\UseMicrotypeSet[protrusion]{basicmath} % disable protrusion for tt fonts
}{}
\IfFileExists{parskip.sty}{%
\usepackage{parskip}
}{% else
\setlength{\parindent}{0pt}
\setlength{\parskip}{6pt plus 2pt minus 1pt}
}
\usepackage{hyperref}
\hypersetup{
            pdftitle={PS3},
            pdfauthor={Soowon Jo},
            pdfborder={0 0 0},
            breaklinks=true}
\urlstyle{same}  % don't use monospace font for urls
\usepackage[margin=1in]{geometry}
\usepackage{color}
\usepackage{fancyvrb}
\newcommand{\VerbBar}{|}
\newcommand{\VERB}{\Verb[commandchars=\\\{\}]}
\DefineVerbatimEnvironment{Highlighting}{Verbatim}{commandchars=\\\{\}}
% Add ',fontsize=\small' for more characters per line
\usepackage{framed}
\definecolor{shadecolor}{RGB}{248,248,248}
\newenvironment{Shaded}{\begin{snugshade}}{\end{snugshade}}
\newcommand{\AlertTok}[1]{\textcolor[rgb]{0.94,0.16,0.16}{#1}}
\newcommand{\AnnotationTok}[1]{\textcolor[rgb]{0.56,0.35,0.01}{\textbf{\textit{#1}}}}
\newcommand{\AttributeTok}[1]{\textcolor[rgb]{0.77,0.63,0.00}{#1}}
\newcommand{\BaseNTok}[1]{\textcolor[rgb]{0.00,0.00,0.81}{#1}}
\newcommand{\BuiltInTok}[1]{#1}
\newcommand{\CharTok}[1]{\textcolor[rgb]{0.31,0.60,0.02}{#1}}
\newcommand{\CommentTok}[1]{\textcolor[rgb]{0.56,0.35,0.01}{\textit{#1}}}
\newcommand{\CommentVarTok}[1]{\textcolor[rgb]{0.56,0.35,0.01}{\textbf{\textit{#1}}}}
\newcommand{\ConstantTok}[1]{\textcolor[rgb]{0.00,0.00,0.00}{#1}}
\newcommand{\ControlFlowTok}[1]{\textcolor[rgb]{0.13,0.29,0.53}{\textbf{#1}}}
\newcommand{\DataTypeTok}[1]{\textcolor[rgb]{0.13,0.29,0.53}{#1}}
\newcommand{\DecValTok}[1]{\textcolor[rgb]{0.00,0.00,0.81}{#1}}
\newcommand{\DocumentationTok}[1]{\textcolor[rgb]{0.56,0.35,0.01}{\textbf{\textit{#1}}}}
\newcommand{\ErrorTok}[1]{\textcolor[rgb]{0.64,0.00,0.00}{\textbf{#1}}}
\newcommand{\ExtensionTok}[1]{#1}
\newcommand{\FloatTok}[1]{\textcolor[rgb]{0.00,0.00,0.81}{#1}}
\newcommand{\FunctionTok}[1]{\textcolor[rgb]{0.00,0.00,0.00}{#1}}
\newcommand{\ImportTok}[1]{#1}
\newcommand{\InformationTok}[1]{\textcolor[rgb]{0.56,0.35,0.01}{\textbf{\textit{#1}}}}
\newcommand{\KeywordTok}[1]{\textcolor[rgb]{0.13,0.29,0.53}{\textbf{#1}}}
\newcommand{\NormalTok}[1]{#1}
\newcommand{\OperatorTok}[1]{\textcolor[rgb]{0.81,0.36,0.00}{\textbf{#1}}}
\newcommand{\OtherTok}[1]{\textcolor[rgb]{0.56,0.35,0.01}{#1}}
\newcommand{\PreprocessorTok}[1]{\textcolor[rgb]{0.56,0.35,0.01}{\textit{#1}}}
\newcommand{\RegionMarkerTok}[1]{#1}
\newcommand{\SpecialCharTok}[1]{\textcolor[rgb]{0.00,0.00,0.00}{#1}}
\newcommand{\SpecialStringTok}[1]{\textcolor[rgb]{0.31,0.60,0.02}{#1}}
\newcommand{\StringTok}[1]{\textcolor[rgb]{0.31,0.60,0.02}{#1}}
\newcommand{\VariableTok}[1]{\textcolor[rgb]{0.00,0.00,0.00}{#1}}
\newcommand{\VerbatimStringTok}[1]{\textcolor[rgb]{0.31,0.60,0.02}{#1}}
\newcommand{\WarningTok}[1]{\textcolor[rgb]{0.56,0.35,0.01}{\textbf{\textit{#1}}}}
\usepackage{graphicx,grffile}
\makeatletter
\def\maxwidth{\ifdim\Gin@nat@width>\linewidth\linewidth\else\Gin@nat@width\fi}
\def\maxheight{\ifdim\Gin@nat@height>\textheight\textheight\else\Gin@nat@height\fi}
\makeatother
% Scale images if necessary, so that they will not overflow the page
% margins by default, and it is still possible to overwrite the defaults
% using explicit options in \includegraphics[width, height, ...]{}
\setkeys{Gin}{width=\maxwidth,height=\maxheight,keepaspectratio}
\setlength{\emergencystretch}{3em}  % prevent overfull lines
\providecommand{\tightlist}{%
  \setlength{\itemsep}{0pt}\setlength{\parskip}{0pt}}
\setcounter{secnumdepth}{0}
% Redefines (sub)paragraphs to behave more like sections
\ifx\paragraph\undefined\else
\let\oldparagraph\paragraph
\renewcommand{\paragraph}[1]{\oldparagraph{#1}\mbox{}}
\fi
\ifx\subparagraph\undefined\else
\let\oldsubparagraph\subparagraph
\renewcommand{\subparagraph}[1]{\oldsubparagraph{#1}\mbox{}}
\fi

% set default figure placement to htbp
\makeatletter
\def\fps@figure{htbp}
\makeatother


\title{PS3}
\author{Soowon Jo}
\date{2/15/2020}

\begin{document}
\maketitle

\#\#Problem Set 3: Trees \& Machines

\begin{verbatim}
## 
## Attaching package: 'dplyr'
\end{verbatim}

\begin{verbatim}
## The following objects are masked from 'package:stats':
## 
##     filter, lag
\end{verbatim}

\begin{verbatim}
## The following objects are masked from 'package:base':
## 
##     intersect, setdiff, setequal, union
\end{verbatim}

\begin{verbatim}
## ── Attaching packages ──────────────────────────────────────────────────────────────────────────────────────────── tidyverse 1.3.0 ──
\end{verbatim}

\begin{verbatim}
## ✓ ggplot2 3.2.1     ✓ purrr   0.3.3
## ✓ tibble  2.1.3     ✓ stringr 1.4.0
## ✓ tidyr   1.0.0     ✓ forcats 0.4.0
## ✓ readr   1.3.1
\end{verbatim}

\begin{verbatim}
## ── Conflicts ─────────────────────────────────────────────────────────────────────────────────────────────── tidyverse_conflicts() ──
## x dplyr::filter() masks stats::filter()
## x dplyr::lag()    masks stats::lag()
\end{verbatim}

\begin{verbatim}
## Loading required package: lattice
\end{verbatim}

\begin{verbatim}
## 
## Attaching package: 'caret'
\end{verbatim}

\begin{verbatim}
## The following object is masked from 'package:purrr':
## 
##     lift
\end{verbatim}

\begin{verbatim}
## Loaded gbm 2.1.5
\end{verbatim}

\begin{verbatim}
## Registered S3 method overwritten by 'tree':
##   method     from
##   print.tree cli
\end{verbatim}

\begin{verbatim}
## randomForest 4.6-14
\end{verbatim}

\begin{verbatim}
## Type rfNews() to see new features/changes/bug fixes.
\end{verbatim}

\begin{verbatim}
## 
## Attaching package: 'randomForest'
\end{verbatim}

\begin{verbatim}
## The following object is masked from 'package:ggplot2':
## 
##     margin
\end{verbatim}

\begin{verbatim}
## The following object is masked from 'package:dplyr':
## 
##     combine
\end{verbatim}

\hypertarget{decision-trees}{%
\subsubsection{Decision Trees}\label{decision-trees}}

\#\#\#\#1. Set up the data and store some things for later use: • Set
seed • Load the data • Store the total number of features minus the
biden feelings in object p • Set λ (shrinkage/learning rate) range from
0.0001 to 0.04, by 0.001

\begin{Shaded}
\begin{Highlighting}[]
\KeywordTok{set.seed}\NormalTok{(}\DecValTok{210}\NormalTok{)}
\KeywordTok{setwd}\NormalTok{(}\StringTok{"/Users/soowonjo/Desktop/MachineLearning/PB3"}\NormalTok{)}
\NormalTok{data =}\StringTok{ }\KeywordTok{read.csv}\NormalTok{(}\StringTok{"nes2008.csv"}\NormalTok{)}
\NormalTok{p <-}\StringTok{ }\KeywordTok{ncol}\NormalTok{(data[}\OperatorTok{-}\DecValTok{1}\NormalTok{]) }
\NormalTok{lambda <-}\KeywordTok{seq}\NormalTok{(}\DataTypeTok{from =} \FloatTok{.0001}\NormalTok{, }\DataTypeTok{to =} \FloatTok{.04}\NormalTok{, }\DataTypeTok{by =} \FloatTok{.001}\NormalTok{)}
\end{Highlighting}
\end{Shaded}

\#\#\#\#2. (10 points) Create a training set consisting of 75\% of the
observations, and a test set with all remaining obs. Note: because you
will be asked to loop over multiple λ values below, these training and
test sets should only be integer values corresponding with row IDs in
the data. This is a little tricky, but think about it carefully. If you
try to set the training and testing sets as before, you will be unable
to loop below.

\begin{Shaded}
\begin{Highlighting}[]
\NormalTok{smp_size <-}\StringTok{ }\KeywordTok{floor}\NormalTok{(}\FloatTok{0.75} \OperatorTok{*}\StringTok{ }\KeywordTok{nrow}\NormalTok{(data))}

\KeywordTok{set.seed}\NormalTok{(}\DecValTok{110}\NormalTok{)}
\NormalTok{train_ind <-}\StringTok{ }\KeywordTok{sample}\NormalTok{(}\KeywordTok{seq_len}\NormalTok{(}\KeywordTok{nrow}\NormalTok{(data)), }\DataTypeTok{size =}\NormalTok{ smp_size)}

\NormalTok{trainset <-}\StringTok{ }\NormalTok{data[train_ind, ]}
\NormalTok{testset <-}\StringTok{ }\NormalTok{data[}\OperatorTok{-}\NormalTok{train_ind, ]}
\end{Highlighting}
\end{Shaded}

\#\#\#\#3. (15 points) Create empty objects to store training and
testing MSE, and then write a loop to perform boosting on the training
set with 1,000 trees for the pre-defined range of values of the
shrinkage parameter, λ. Then, plot the training set and test set MSE
across shrinkage values.

\begin{Shaded}
\begin{Highlighting}[]
\NormalTok{testMSE <-}\StringTok{ }\KeywordTok{vector}\NormalTok{(}\DataTypeTok{mode =} \StringTok{"numeric"}\NormalTok{, }\DataTypeTok{length =} \KeywordTok{length}\NormalTok{(lambda))}
\NormalTok{trainingMSE <-}\StringTok{ }\KeywordTok{vector}\NormalTok{(}\DataTypeTok{mode =} \StringTok{"numeric"}\NormalTok{, }\DataTypeTok{length =} \KeywordTok{length}\NormalTok{(lambda))}

\ControlFlowTok{for}\NormalTok{(i }\ControlFlowTok{in} \KeywordTok{seq_along}\NormalTok{(lambda)) \{}
\CommentTok{# boosting training set}
\NormalTok{boost.train <-}\StringTok{ }\KeywordTok{gbm}\NormalTok{(biden }\OperatorTok{~}\NormalTok{.,}
    \DataTypeTok{data =}\NormalTok{ trainset,}
    \DataTypeTok{distribution =} \StringTok{"gaussian"}\NormalTok{,}
    \DataTypeTok{n.trees =} \DecValTok{1000}\NormalTok{,}
    \DataTypeTok{shrinkage =}\NormalTok{ lambda[i],}
    \DataTypeTok{interaction.depth =} \DecValTok{4}\NormalTok{)}

\NormalTok{training.pred <-}\StringTok{ }\KeywordTok{predict}\NormalTok{(boost.train, }\DataTypeTok{newdata =}\NormalTok{ trainset, }\DataTypeTok{n.trees =} \DecValTok{1000}\NormalTok{)}
\NormalTok{training.mse <-}\StringTok{ }\NormalTok{Metrics}\OperatorTok{::}\KeywordTok{mse}\NormalTok{(training.pred, trainset}\OperatorTok{$}\NormalTok{biden)}

\CommentTok{# making prediction on the test set}
\NormalTok{test.pred <-}\StringTok{ }\KeywordTok{predict}\NormalTok{(boost.train, }\DataTypeTok{newdata =}\NormalTok{ testset, }\DataTypeTok{n.trees =} \DecValTok{1000}\NormalTok{)}
\NormalTok{test.mse <-}\StringTok{ }\NormalTok{Metrics}\OperatorTok{::}\KeywordTok{mse}\NormalTok{(test.pred, testset}\OperatorTok{$}\NormalTok{biden) }

\CommentTok{# extract MSE and lambda values}
\NormalTok{trainingMSE[i] <-}\StringTok{ }\NormalTok{training.mse}
\NormalTok{testMSE[i] <-}\StringTok{ }\NormalTok{test.mse}

\NormalTok{result <-}\StringTok{ }\KeywordTok{cbind}\NormalTok{(lambda, trainingMSE, testMSE)}
\NormalTok{result <-}\StringTok{ }\NormalTok{result }\OperatorTok
\StringTok{    }\KeywordTok{as.tibble}\NormalTok{()}
\NormalTok{\}}
\end{Highlighting}
\end{Shaded}

\begin{verbatim}
## Warning: `as.tibble()` is deprecated, use `as_tibble()` (but mind the new semantics).
## This warning is displayed once per session.
\end{verbatim}

\begin{Shaded}
\begin{Highlighting}[]
\NormalTok{result }\OperatorTok
\StringTok{    }\KeywordTok{ggplot}\NormalTok{(}\KeywordTok{aes}\NormalTok{(}\DataTypeTok{x =}\NormalTok{ lambda)) }\OperatorTok{+}
\StringTok{    }\KeywordTok{geom_point}\NormalTok{(}\KeywordTok{aes}\NormalTok{(}\DataTypeTok{y =}\NormalTok{ trainingMSE, }\DataTypeTok{color =} \StringTok{"Training Set"}\NormalTok{)) }\OperatorTok{+}
\StringTok{    }\KeywordTok{geom_point}\NormalTok{(}\KeywordTok{aes}\NormalTok{(}\DataTypeTok{y =}\NormalTok{ testMSE, }\DataTypeTok{color =} \StringTok{"Test Set"}\NormalTok{)) }\OperatorTok{+}
\StringTok{    }\KeywordTok{geom_line}\NormalTok{(}\KeywordTok{aes}\NormalTok{(}\DataTypeTok{y =}\NormalTok{ trainingMSE, }\DataTypeTok{color =} \StringTok{"Training Set"}\NormalTok{)) }\OperatorTok{+}
\StringTok{    }\KeywordTok{geom_line}\NormalTok{(}\KeywordTok{aes}\NormalTok{(}\DataTypeTok{y =}\NormalTok{ testMSE, }\DataTypeTok{color =} \StringTok{"Test Set"}\NormalTok{)) }\OperatorTok{+}
\StringTok{    }\KeywordTok{labs}\NormalTok{(}\DataTypeTok{x =} \StringTok{"Shrinkage Values (Lambda)"}\NormalTok{, }\DataTypeTok{y =} \StringTok{"MSE values"}\NormalTok{)}
\end{Highlighting}
\end{Shaded}

\includegraphics{PS3_files/figure-latex/unnamed-chunk-4-1.pdf}

\#\#\#\#4. (10 points) The test MSE values are insensitive to some
precise value of λ as long as its small enough. Update the boosting
procedure by setting λ equal to 0.01 (but still over 1000 trees). Report
the test MSE and discuss the results. How do they compare?

\begin{Shaded}
\begin{Highlighting}[]
\NormalTok{boost.train <-}\StringTok{ }\KeywordTok{gbm}\NormalTok{(biden }\OperatorTok{~}\NormalTok{.,}
    \DataTypeTok{data =}\NormalTok{ trainset,}
    \DataTypeTok{distribution =} \StringTok{"gaussian"}\NormalTok{,}
    \DataTypeTok{n.trees =} \DecValTok{1000}\NormalTok{,}
    \DataTypeTok{shrinkage =} \FloatTok{0.01}\NormalTok{,}
    \DataTypeTok{interaction.depth =} \DecValTok{4}\NormalTok{)}

\CommentTok{# making prediction on the test set}
\NormalTok{test.pred_new <-}\StringTok{ }\KeywordTok{predict}\NormalTok{(boost.train, }\DataTypeTok{newdata =}\NormalTok{ testset, }\DataTypeTok{n.trees =} \DecValTok{1000}\NormalTok{)}
\NormalTok{test.mse_new <-}\StringTok{ }\NormalTok{Metrics}\OperatorTok{::}\KeywordTok{mse}\NormalTok{(test.pred_new, testset}\OperatorTok{$}\NormalTok{biden) }

\CommentTok{# report the test MSE}
\NormalTok{test.mse_new}
\end{Highlighting}
\end{Shaded}

\begin{verbatim}
## [1] 388.9172
\end{verbatim}

\begin{Shaded}
\begin{Highlighting}[]
\NormalTok{result }\OperatorTok
\StringTok{    }\KeywordTok{ggplot}\NormalTok{(}\KeywordTok{aes}\NormalTok{(}\DataTypeTok{x =}\NormalTok{ lambda)) }\OperatorTok{+}
\StringTok{    }\KeywordTok{geom_point}\NormalTok{(}\KeywordTok{aes}\NormalTok{(}\DataTypeTok{y =}\NormalTok{ trainingMSE, }\DataTypeTok{color =} \StringTok{"Training Set"}\NormalTok{)) }\OperatorTok{+}
\StringTok{    }\KeywordTok{geom_point}\NormalTok{(}\KeywordTok{aes}\NormalTok{(}\DataTypeTok{y =}\NormalTok{ testMSE, }\DataTypeTok{color =} \StringTok{"Test Set"}\NormalTok{)) }\OperatorTok{+}
\StringTok{    }\KeywordTok{geom_line}\NormalTok{(}\KeywordTok{aes}\NormalTok{(}\DataTypeTok{y =}\NormalTok{ trainingMSE, }\DataTypeTok{color =} \StringTok{"Training Set"}\NormalTok{)) }\OperatorTok{+}
\StringTok{    }\KeywordTok{geom_line}\NormalTok{(}\KeywordTok{aes}\NormalTok{(}\DataTypeTok{y =}\NormalTok{ testMSE, }\DataTypeTok{color =} \StringTok{"Test Set"}\NormalTok{)) }\OperatorTok{+}
\StringTok{    }\KeywordTok{geom_vline}\NormalTok{(}\DataTypeTok{xintercept=}\FloatTok{0.01}\NormalTok{, }\DataTypeTok{linetype=}\StringTok{"dashed"}\NormalTok{)}\OperatorTok{+}
\StringTok{    }\KeywordTok{labs}\NormalTok{(}\DataTypeTok{x =} \StringTok{"Shrinkage Values (Lambda)"}\NormalTok{, }\DataTypeTok{y =} \StringTok{"MSE values"}\NormalTok{)}
\end{Highlighting}
\end{Shaded}

\includegraphics{PS3_files/figure-latex/unnamed-chunk-5-1.pdf} The test
MSE for λ equal to 0.01 is 388.9172. Based on the plot above, the MSE
appears roughly the same as λ increases. When the value of λ is smaller
than 0.01, the models have much larger test MSEs, which means that the
model test MSE is insensitive to shrinkage values once they are larger
than 0.01.

\#\#\#\#5. (10 points) Now apply bagging to the training set. What is
the test set MSE for this approach?

\begin{Shaded}
\begin{Highlighting}[]
\NormalTok{bagging <-}\StringTok{ }\KeywordTok{randomForest}\NormalTok{(biden }\OperatorTok{~}\StringTok{ }\NormalTok{.,}
    \DataTypeTok{data =}\NormalTok{ trainset, }
    \DataTypeTok{mtry =}\NormalTok{ p) }

\NormalTok{bagging.test.pred <-}\StringTok{ }\KeywordTok{predict}\NormalTok{(bagging, }\DataTypeTok{newdata =}\NormalTok{ testset)}
\NormalTok{bagging.test.mse <-}\StringTok{ }\NormalTok{Metrics}\OperatorTok{::}\KeywordTok{mse}\NormalTok{(bagging.test.pred, testset}\OperatorTok{$}\NormalTok{biden)}
\NormalTok{bagging.test.mse}
\end{Highlighting}
\end{Shaded}

\begin{verbatim}
## [1] 451.3808
\end{verbatim}

\#\#\#\#6. (10 points) Now apply random forest to the training set. What
is the test set MSE for this approach?

\begin{Shaded}
\begin{Highlighting}[]
\NormalTok{rf <-}\StringTok{ }\KeywordTok{randomForest}\NormalTok{(biden }\OperatorTok{~}\StringTok{ }\NormalTok{., }
                \DataTypeTok{data =}\NormalTok{ trainset, }
                \DataTypeTok{importance =} \OtherTok{TRUE}\NormalTok{) }

\NormalTok{rf.test.pred <-}\StringTok{ }\KeywordTok{predict}\NormalTok{(rf, }\DataTypeTok{newdata =}\NormalTok{ testset)}
\NormalTok{rf.test.mse <-}\StringTok{ }\NormalTok{Metrics}\OperatorTok{::}\KeywordTok{mse}\NormalTok{(rf.test.pred, testset}\OperatorTok{$}\NormalTok{biden)}
\NormalTok{rf.test.mse}
\end{Highlighting}
\end{Shaded}

\begin{verbatim}
## [1] 402.4014
\end{verbatim}

\#\#\#\#7. (5 points) Now apply linear regression to the training set.
What is the test set MSE for this approach?

\begin{Shaded}
\begin{Highlighting}[]
\NormalTok{lm_train <-}\StringTok{ }\KeywordTok{glm}\NormalTok{(biden }\OperatorTok{~}\StringTok{ }\NormalTok{.,}\DataTypeTok{data =}\NormalTok{ trainset)}
\NormalTok{lm.test.pred <-}\StringTok{ }\KeywordTok{predict}\NormalTok{(lm_train, }\DataTypeTok{newdata =}\NormalTok{ testset)}
\NormalTok{lm.test.mse <-}\StringTok{ }\NormalTok{Metrics}\OperatorTok{::}\KeywordTok{mse}\NormalTok{(lm.test.pred, testset}\OperatorTok{$}\NormalTok{biden)}
\NormalTok{lm.test.mse}
\end{Highlighting}
\end{Shaded}

\begin{verbatim}
## [1] 392.3925
\end{verbatim}

\#\#\#\#8. (5 points) Compare test errors across all fits. Discuss which
approach generally fits best and how you concluded this. The boosting
approach results in the lowest MSE across all fits (388.9172) meaning
that this approach fits best. The test MSEs received from the boosting
and linear regression approaches were similar around 390. The fit with
the bagging approach with a lambda parameter of at least 0.01, on the
other hand, has the largest test error (451.3808) making it the worst
fit of all methods. The bagging approach has an ``mtry'' values set to p
= 5. The mtry value in the bagging approach is larger than that in the
randomForest function (p/3). When comparing the error values of both
randomforest and bagging approaches, this implies that an increased mtry
value results in an increased mean squared error.

\hypertarget{support-vector-machines}{%
\subsubsection{Support Vector Machines}\label{support-vector-machines}}

\#\#\#\#1. Create a training set with a random sample of size 800, and a
test set containing the remaining observations.

\begin{Shaded}
\begin{Highlighting}[]
\KeywordTok{set.seed}\NormalTok{(}\DecValTok{100}\NormalTok{)}
\NormalTok{train=}\KeywordTok{sample}\NormalTok{(}\KeywordTok{nrow}\NormalTok{(OJ),}\DecValTok{800}\NormalTok{)}
\NormalTok{OJ.train =}\StringTok{ }\NormalTok{OJ[train,]}
\NormalTok{OJ.test =}\StringTok{ }\NormalTok{OJ[}\OperatorTok{-}\NormalTok{train,]}
\end{Highlighting}
\end{Shaded}

\#\#\#\#2. (10 points) Fit a support vector classifier to the training
data with cost = 0.01, with Purchase as the response and all other
features as predictors. Discuss the results.

\begin{Shaded}
\begin{Highlighting}[]
\NormalTok{svmfit <-}\StringTok{ }\KeywordTok{svm}\NormalTok{(Purchase }\OperatorTok{~}\StringTok{ }\NormalTok{., }
             \DataTypeTok{data =}\NormalTok{ OJ.train, }
             \DataTypeTok{kernel =} \StringTok{"linear"}\NormalTok{, }
             \DataTypeTok{cost =} \FloatTok{0.01}\NormalTok{, }
             \DataTypeTok{scale =} \OtherTok{FALSE}\NormalTok{)}

\KeywordTok{summary}\NormalTok{(svmfit)}
\end{Highlighting}
\end{Shaded}

\begin{verbatim}
## 
## Call:
## svm(formula = Purchase ~ ., data = OJ.train, kernel = "linear", cost = 0.01, 
##     scale = FALSE)
## 
## 
## Parameters:
##    SVM-Type:  C-classification 
##  SVM-Kernel:  linear 
##        cost:  0.01 
## 
## Number of Support Vectors:  623
## 
##  ( 312 311 )
## 
## 
## Number of Classes:  2 
## 
## Levels: 
##  CH MM
\end{verbatim}

The result indicates that more support vectors (623) are required to get
the proper separation of data when the data is enlarged into a higher
dimensional space. Moreover, it requires two classes and levels.

\#\#\#\#3. (5 points) Display the confusion matrix for the
classification solution, and also report both the training and test set
error rates.

\begin{Shaded}
\begin{Highlighting}[]
\CommentTok{#Generate predicted values for train and test sets}
\NormalTok{Purchase1 <-}\StringTok{ }\KeywordTok{predict}\NormalTok{(svmfit, OJ.train)}
\NormalTok{Purchase2 <-}\StringTok{ }\KeywordTok{predict}\NormalTok{(svmfit, OJ.test)}

\CommentTok{#Confusion matrix for training set predictions}
\KeywordTok{table}\NormalTok{(}\DataTypeTok{predicted =}\NormalTok{ Purchase1, }
      \DataTypeTok{true =}\NormalTok{ OJ.train}\OperatorTok{$}\NormalTok{Purchase)}
\end{Highlighting}
\end{Shaded}

\begin{verbatim}
##          true
## predicted  CH  MM
##        CH 466 177
##        MM  22 135
\end{verbatim}

\begin{Shaded}
\begin{Highlighting}[]
\CommentTok{#Error rate, calculated as misclassified/total}
\NormalTok{(}\DecValTok{177}\OperatorTok{+}\DecValTok{22}\NormalTok{)}\OperatorTok{/}\DecValTok{800}
\end{Highlighting}
\end{Shaded}

\begin{verbatim}
## [1] 0.24875
\end{verbatim}

\begin{Shaded}
\begin{Highlighting}[]
\CommentTok{#Confusion matrix for test set predictions}
\KeywordTok{table}\NormalTok{(}\DataTypeTok{predicted =}\NormalTok{ Purchase2, }
      \DataTypeTok{true =}\NormalTok{ OJ.test}\OperatorTok{$}\NormalTok{Purchase)}
\end{Highlighting}
\end{Shaded}

\begin{verbatim}
##          true
## predicted  CH  MM
##        CH 157  63
##        MM   8  42
\end{verbatim}

\begin{Shaded}
\begin{Highlighting}[]
\CommentTok{#Error rate, calculated as misclassified/total}
\NormalTok{(}\DecValTok{63}\OperatorTok{+}\DecValTok{8}\NormalTok{)}\OperatorTok{/}\DecValTok{270}
\end{Highlighting}
\end{Shaded}

\begin{verbatim}
## [1] 0.262963
\end{verbatim}

\#\#\#\#4. (10 points) Find an optimal cost in the range of 0.01 to 1000
(specific range values can vary; there is no set vector of range values
you must use).

\begin{Shaded}
\begin{Highlighting}[]
\NormalTok{tune_c <-}\StringTok{ }\KeywordTok{tune}\NormalTok{(svm, }
\NormalTok{                Purchase }\OperatorTok{~}\StringTok{ }\NormalTok{., }
                \DataTypeTok{data =}\NormalTok{ OJ.train, }
                \DataTypeTok{kernel =} \StringTok{"linear"}\NormalTok{, }
                \DataTypeTok{ranges =} \KeywordTok{list}\NormalTok{(}\DataTypeTok{cost =} \KeywordTok{c}\NormalTok{(}\FloatTok{0.01}\NormalTok{, }\FloatTok{0.1}\NormalTok{, }\DecValTok{1}\NormalTok{, }\DecValTok{10}\NormalTok{, }\DecValTok{100}\NormalTok{, }\DecValTok{1000}\NormalTok{)))}

\CommentTok{#Subset best model and look at summary to identify its cost value}
\NormalTok{tuned_model <-}\StringTok{ }\NormalTok{tune_c}\OperatorTok{$}\NormalTok{best.model}
\KeywordTok{summary}\NormalTok{(tuned_model)}
\end{Highlighting}
\end{Shaded}

\begin{verbatim}
## 
## Call:
## best.tune(method = svm, train.x = Purchase ~ ., data = OJ.train, 
##     ranges = list(cost = c(0.01, 0.1, 1, 10, 100, 1000)), kernel = "linear")
## 
## 
## Parameters:
##    SVM-Type:  C-classification 
##  SVM-Kernel:  linear 
##        cost:  1 
## 
## Number of Support Vectors:  325
## 
##  ( 163 162 )
## 
## 
## Number of Classes:  2 
## 
## Levels: 
##  CH MM
\end{verbatim}

\#\#\#\#5. (10 points) Compute the optimal training and test error rates
using this new value for cost. Display the confusion matrix for the
classification solution, and also report both the training and test set
error rates. How do the error rates compare? Discuss the results in
substantive terms (e.g., how well did your optimally tuned classifer
perform? etc.)

\begin{Shaded}
\begin{Highlighting}[]
\CommentTok{#Generate predicted values for train and test sets}
\NormalTok{Purchase3 <-}\StringTok{ }\KeywordTok{predict}\NormalTok{(tuned_model, OJ.train)}
\NormalTok{Purchase4 <-}\StringTok{ }\KeywordTok{predict}\NormalTok{(tuned_model, OJ.test)}

\CommentTok{#Confusion matrix for training set predictions}
\NormalTok{t1 <-}\StringTok{ }\KeywordTok{table}\NormalTok{(}\DataTypeTok{predicted =}\NormalTok{ Purchase3, }
      \DataTypeTok{true =}\NormalTok{ OJ.train}\OperatorTok{$}\NormalTok{Purchase)}

\CommentTok{#Error rate, calculated as misclassified / total}
\NormalTok{(t1[}\DecValTok{1}\NormalTok{,}\DecValTok{2}\NormalTok{]}\OperatorTok{+}\NormalTok{t1[}\DecValTok{2}\NormalTok{,}\DecValTok{1}\NormalTok{])}\OperatorTok{/}\DecValTok{800}
\end{Highlighting}
\end{Shaded}

\begin{verbatim}
## [1] 0.1575
\end{verbatim}

\begin{Shaded}
\begin{Highlighting}[]
\CommentTok{#Confusion matrix for test set predictions}
\NormalTok{t2 <-}\StringTok{ }\KeywordTok{table}\NormalTok{(}\DataTypeTok{predicted =}\NormalTok{ Purchase4, }
      \DataTypeTok{true =}\NormalTok{ OJ.test}\OperatorTok{$}\NormalTok{Purchase)}

\CommentTok{#Error rate, calculated as misclassified / total}
\NormalTok{(t2[}\DecValTok{1}\NormalTok{,}\DecValTok{2}\NormalTok{]}\OperatorTok{+}\NormalTok{t2[}\DecValTok{2}\NormalTok{,}\DecValTok{1}\NormalTok{])}\OperatorTok{/}\DecValTok{270}
\end{Highlighting}
\end{Shaded}

\begin{verbatim}
## [1] 0.1814815
\end{verbatim}

The optimal test error rates for training and test sets are 0.1575 and
0.1815, respectively for the tuned model. The test error rates for
training and test sets are 0.2487 and 0.2629, respectively for the
original model. This result shows that the tuned model has lower test
error rates for all sets than the original model. This would have been
possible since the tuning function determines the cost value which
results in the most accurate classifier.

\end{document}
